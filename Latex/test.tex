\documentclass{report}

\usepackage[default]{sourcesanspro}
\usepackage[T1]{fontenc}
\usepackage{titlesec}
\usepackage{xcolor}
\usepackage{graphicx}
\usepackage{fancyhdr}
\usepackage{everypage}
\usepackage{soul} % For letter spacing and background color
\usepackage[letterspace=300]{microtype} 
\usepackage{background}
\usepackage{lipsum}% for some dummy text

\usepackage[margin=2cm]{geometry}
\usepackage[ngerman]{babel}


\backgroundsetup{
  scale=1,
  color=white,
  opacity=0,
  angle=0,
  position=current page.center,
  vshift=0cm,
  hshift=0cm,
  contents={}
}

\definecolor{customgreen}{HTML}{89B19B}
\definecolor{customblue}{HTML}{8FA0D4}



\DeclareRobustCommand{\ebseries}{\fontseries{eb}\selectfont}
\DeclareTextFontCommand{\texteb}{\ebseries}

\DeclareRobustCommand{\sbseries}{\fontseries{sb}\selectfont}
\DeclareTextFontCommand{\textsb}{\sbseries}

\DeclareRobustCommand{\elseries}{\fontseries{el}\selectfont}
\DeclareTextFontCommand{\textel}{\elseries}

\DeclareRobustCommand{\lseries}{\fontseries{l}\selectfont}
\DeclareTextFontCommand{\textl}{\lseries}

\renewcommand{\seriesdefault}{l} % Set default series to light

\titleformat{\chapter}
  {\color{white}\sffamily\sbseries\fontsize{22}{26}\selectfont } % Format of the chapter title
    {\thechapter.} % Chapter number
    {10pt} % Space between chapter number and title
    {} % Format of the chapter title text

\titlespacing*{\chapter}{0pt}{-18pt}{8pt} % Adjust the spacing

% Definition des ersten Hintergrunds
\newcommand{\backgroundOne}[1]{
  \backgroundsetup{
    scale=1,
    color=red,
    opacity=1,
    angle=0,
    position=current page.north,
    vshift=-2cm,
    hshift=0cm,
    contents={\begin{tikzpicture}[remember picture,overlay]
      \fill[color=#1] (current page.north west) rectangle ([yshift=-4.78cm]current page.north east);
    \end{tikzpicture}}
  }
}

% Definition des zweiten Hintergrunds
\newcommand{\backgroundTwo}[1]{
  \backgroundsetup{
    scale=1,
    color=blue,
    opacity=1,
    angle=0,
    position=current page.north,
    vshift=-2cm,
    hshift=0cm,
    contents={\begin{tikzpicture}[remember picture,overlay]
      \fill[color=#1] (current page.north west) rectangle ([yshift=-1.76cm]current page.north east);
    \end{tikzpicture}}
  }
}



% Define the chapter start page style with color
\newcommand{\chapterstartpage}[3]{
    \newpage % Start a new page
    \chapter{#1}
    % \addcontentsline{toc}{chapter}{#1} % Add the chapter title to the table of contents
    \backgroundOne{#3} % Erster Hintergrund
    \BgThispage % Hintergrund auf dieser Seite anwenden
    
    \color{white} % Set text color to white
    \noindent\rule{\textwidth}{0.8pt} % White line 0.8pt thick
    \newline\vspace{0pt} % Space between line and subtitle
    \newline\sffamily{\bfseries\fontsize{7}{9}\selectfont\textls[300]{#2}} % Subtitle in bold, size 7, all caps, letter spacing 30%
    \newline\vspace{10pt} % Space after subtitle
    \color{black} % Set text color back to black
  \newpage
  \backgroundTwo{#3} % Zweiter Hintergrund
 % Hintergrund auf dieser Seite anwenden
% Entfernen des Hooks
}




\begin{document}

\begin{titlepage}
    \centering
    \vspace*{2cm}
    \includegraphics[width=0.15\textwidth]{example-image-1x1}\par\vspace{1cm}
    {\scshape\LARGE Columbidae University \par}
    \vspace{1cm}
    {\scshape\Large Final year project\par}
    \vspace{1.5cm}
    {\huge\bfseries Pigeons love doves\par}
    \vspace{2cm}
    {\Large\itshape John Birdwatch\par}
    \vfill
    supervised by\par
    Dr.~Mark \textsc{Brown}

    \vfill
\end{titlepage}

\tableofcontents

\chapter{Introduction}

\section{Lorem ipsum dolor sit amet}
\lipsum[2]

{\sffamily\elseries extra-light text}

{\sffamily\lseries light text}

{\sffamily regular text}

{\sffamily\sbseries semi-bold text}

{\sffamily\bfseries bold text}

\chapterstartpage{Anleitung}{HIER FINDEST DU EINE DETAILLIERTE ANLEITUNG ZU ALLEN FUNKTIONEN VON MATH-NODES}{customgreen}

\section{Lorem ipsum dolor sit amet}
loren ipsum dolor sit amet loren ipsum dolor sit amet loren ipsum dolor sit amet loren ipsum dolor sit amet loren ipsum dolor sit amet loren ipsum dolor sit amet loren ipsum dolor sit amet loren ipsum dolor sit amet loren ipsum dolor sit amet loren ipsum dolor sit amet loren ipsum dolor sit amet loren ipsum dolor sit amet loren ipsum dolor sit amet loren ipsum dolor sit amet loren ipsum dolor sit amet loren ipsum dolor sit amet 

\newpage

\section{Lorem ipsum dolor sit amet}
\lipsum[2]

\chapterstartpage{Test}{HIER FINDEST DU EINE DETAILLIERTE ANLEITUNG ZU ALLEN FUNKTIONEN VON MATH-NODES}{customblue}
\lipsum[4]

\newpage

\lipsum[4]
\newpage

\lipsum[4]
\end{document}